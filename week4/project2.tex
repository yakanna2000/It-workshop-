\documentclass{article}
\usepackage{xcolor}
\usepackage{graphicx}
\usepackage{amsmath}
\title{THE 18.821 MATHEMATICS PROJECT LAB REPORT[REPLACE THIS WITH YOUR OWN SHORT DESCRIPTIVE TITLE!]}
\author{X. BURPS, P. GURPS}
\begin{document}
	\maketitle
	\abstractname .This is a LATEX template for 18.821, which you can use  for your own reports.
	\vspace{0.001cm}
	\centering
	\section{\small INTRODUCTION}
	\vspace{0.001cm}
	This brief document shows some examples of the use of LATEX and
	indicates some special features of the Math Lab report style. The
	\textcolor{blue}{course website} contains links to several LATEX manuals.\\
	End the introduction by describing the contents of the paper sec­
	tion by section, and which team member(s) wrote each of them. For
	instance, Section \textcolor{blue}{6} discusses referencing, and is written by P. Gurps.
	\vspace{0.001cm}
	\centering
	\section{\small LATEX EXAMPLES}
	\vspace{0.001cm}
	Here are some ways of producing mathematical symbols. Some are
	package which pre-defined either in LATEX or in the AMS$\sum_{i=1}^{n}$1=n,$\int_{0}^{n}$xdx=$\frac{n^2}{2}$
	We’ve defined a few other symbols at the start of the document, for
	instance  .You can make marginal notes for yourself or your
	co-authors like this:\hspace{10cm}
	Unfinished here?
	If you want to typeset equations, there are many choices, with or
	without numbering:
	\begin{equation}
		\int_{0}^{1}xdx=1/2
	\end{equation}
or
	\begin{equation}
		\sum_{i=1}^{0}
	\end{equation}
or
	\begin{equation}
		1 - 1 + 1 - ...= \frac{1}{2}
	\end{equation}
	$\overline{Date: April}$18 2024
	\vspace{0.1cm}
	\vskip 0.01mm
	X. BURPS, P. GURPS
\begin{figure}[htbp]
	\centering
	\includegraphics[width=0.5\textwidth]{/home/user/Downloads/image1.jpeg}
	\caption{My first .pdf figure}\label{FIGURE:1}
\end{figure}\\
	If you want a number for an equation, do it like this:
	\begin{equation}
		\lim_{n->1}\sum_{k=1}^{n}\frac{1}{k^2}=\frac{\pi}{6}
	\end{equation}
	This can then be referred to as (\textcolor{blue}{1}), which is much easier than keeping
	track of numbers by hand. To group several equations, aligning on the
	= sign, do it like this:\\
	\begin{equation}
		x1 + 2x2 + 3x3= 7\\
		y = mx + c\\
		= 4x − 9.
	\end{equation}
You can easily embed hyperlinks into the output .pdf document:
\textcolor{blue}{click} here for example.
	\centering
	\section{\small IMAGES}
	Figure \textcolor{blue}{1} is an example of a .pdf image put into a floating environ­
	ment, which means LaTeX will draw it wherever there’s enough space
	left in your manuscript. Look at the .tex original to see how to insert
	a figure like this.
	\section{\small THEOREMS AND SUCH}
	An example of a “conjecture environment” is given below, in Con­
	jecture \textcolor{blue}{4.1}.Theorems, lemmas, propositions, definitions, and such all
	use the same command with the appropriate name changed. In fact,
	\centering\\
	THE 18.821 REPORT\\
	\vspace{0.5cm}
	if you look at the top of this .tex file, you can see where we’ve defined
	these environments.
	\subsection{Conjecture} \textit{(Vaught’s Conjecture). Let T be a countable com­
		plete theory. If T has fewer than 2ℵ0 many countable models (up to
		isomorphism), then it has countably many countable models.}
	\subsection{Theorem}When it rains it pours.\\
	Proof. Well, yes.
	\section{\small FILE TYPES USED BY LATEX}
	You will write your text as a .tex file using any text editor (though
	WYSIWYG editors are troublesome). Traditionally one then runs
	LATEX and obtains a .dvi file, which can be viewed on the screen using a
	dvi viewer. To include images, and then prepare the file for printing or
	submission, one typically translates the .dvi into either .ps (Postscript)
	or .pdf (Adobe PDF).\\
	Your report will be submitted as a .pdf document.\texttt{pdf latex} command produces a .pdf file directly from a .tex file. This command
	works well with included .pdf files, but does not handle .eps files.
	An .eps file can be converted to a .pdf file by viewing it and saving
	as a .pdf file, or by\texttt{ps2pdf filename.eps,}which produces \texttt{filename.pdf.}Under MikTeX with WinEdt, all necessary commands
	will appear under “Accessories” in the WinEdt menu.\\
	Finally, Matlab can be made to produce .eps files by typing\\
	\texttt{print -deps filename} at the prompt.
	\section{\small QUOTING SOURCES}
	In your work, keep notes of the literature you’ve used, including
	websites. Cite the references you use; failure to do so constitutes pla­giarism. Every bibliography item should be referenced somewhere in
	the paper. Quote as precisely as possible: [\textcolor{blue}{1} pages 76–78] rather than
	[\textcolor{blue}{1}]. [\textcolor{blue}{2}] was a useful background reference, too.
	\centering
	\cite{Pgurps123hello},
	\cite{Xburps123hello}
	\bibliographystyle{plain}
	\bibliography{ref2.bib}
	\appendix APPENDIX\\
	Appendices are useful for putting in code or data.\\
	\vspace{2mm}
	MIT OpenCourseWare\\
	\textcolor{blue}{\underline{http://ocw.mit.edu}}\\
	\vspace{2cm}
	\large 18.821 Project Laboratory in Mathematics\\
	Spring 2013\\
	\vspace{2cm}
	For information about citing these materials or our Terms of Use, visit:\textcolor{blue}{\underline{http://ocw.mit.edu/terms.}}
\end{document}










